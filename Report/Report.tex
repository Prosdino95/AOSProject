%%%%%%%%%%%%%%%%%%%%%%%%%%%%%%%%%%%%%%%%%
% University Assignment Title Page 
% LaTeX Template
% Version 1.0 (27/12/12)
%
% This template has been downloaded from:
% http://www.LaTeXTemplates.com
%
% Original author:
% WikiBooks (http://en.wikibooks.org/wiki/LaTeX/Title_Creation)
%
% License:
% CC BY-NC-SA 3.0 (http://creativecommons.org/licenses/by-nc-sa/3.0/)
% 
% Instructions for using this template:
% This title page is capable of being compiled as is. This is not useful for 
% including it in another document. To do this, you have two options: 
%
% 1) Copy/paste everything between \begin{document} and \end{document} 
% starting at \begin{titlepage} and paste this into another LaTeX file where you 
% want your title page.
% OR
% 2) Remove everything outside the \begin{titlepage} and \end{titlepage} and 
% move this file to the same directory as the LaTeX file you wish to add it to. 
% Then add \input{./title_page_1.tex} to your LaTeX file where you want your
% title page.
%
%%%%%%%%%%%%%%%%%%%%%%%%%%%%%%%%%%%%%%%%%
%\title{Title page with logo}
%----------------------------------------------------------------------------------------
%	PACKAGES AND OTHER DOCUMENT CONFIGURATIONS
%----------------------------------------------------------------------------------------

\documentclass[12pt]{article}
\usepackage[english]{babel}
\usepackage[utf8x]{inputenc}
\usepackage{amsmath}
\usepackage{graphicx}
\usepackage{listings}
\usepackage[colorinlistoftodos]{todonotes}

\begin{document}

\begin{titlepage}

\newcommand{\HRule}{\rule{\linewidth}{0.5mm}} % Defines a new command for the horizontal lines, change thickness here

\center % Center everything on the page
 
%----------------------------------------------------------------------------------------
%	HEADING SECTIONS
%----------------------------------------------------------------------------------------

\textsc{\LARGE Politenico di Milano}\\[1.5cm] % Name of your university/college
\textsc{\Large Dipartimento Elettronica, Informazione e Bioingegneria}\\[0.5cm] % Major heading such as course name
\textsc{\large HEAPLab Project Report}\\[0.5cm] % Minor heading such as course title

%----------------------------------------------------------------------------------------
%	TITLE SECTION
%----------------------------------------------------------------------------------------

\HRule \\[0.4cm]
{ \huge \bfseries Performance Estimation for TAFFO via LLVM-MCA}\\[0.4cm] % Title of your document
\HRule \\[1.5cm]
 
%----------------------------------------------------------------------------------------
%	AUTHOR SECTION
%----------------------------------------------------------------------------------------

\begin{minipage}{0.4\textwidth}
\begin{flushleft} \large
\emph{Author:}\\
Marco \textsc{Prosdocimi} % Your name
\end{flushleft}
\end{minipage}
~
\begin{minipage}{0.4\textwidth}
\begin{flushright} \large
\emph{Supervisors:} \\
Stefano \textsc{Cherubin} % Supervisor's Name
Daniele \textsc{Cattaneo} % Supervisor's Name
\end{flushright}
\end{minipage}\\[2cm]

% If you don't want a supervisor, uncomment the two lines below and remove the section above
%\Large \emph{Author:}\\
%John \textsc{Smith}\\[3cm] % Your name

%----------------------------------------------------------------------------------------
%	DATE SECTION
%----------------------------------------------------------------------------------------

{\large \today}\\[2cm] % Date, change the \today to a set date if you want to be precise

%----------------------------------------------------------------------------------------
%	LOGO SECTION
%----------------------------------------------------------------------------------------

\includegraphics[width=100pt]{heaplogo.pdf}\\[1cm] % Include a department/university logo - this will require the graphicx package
 
%----------------------------------------------------------------------------------------

\vfill % Fill the rest of the page with whitespace

\end{titlepage}




\begin{abstract}

Use LLVM-MCA to compare the fixed-point code produced by TAFFO with the original floating-point code for all loops in the code. 
LLVM-MCA is a tool that simulates the inner behavior of the CPU to estimate the performance of a machine code snippet.
TAFFO is an autotuning framework, based on LLVM 8, which tries to replace floating-point operations with fixed-point operations as much as possible.

\end{abstract}

\section{Introduction}

The purpose of the project is to write a tool to find body loops in a c program, compile them with and without TAFFO, and use LLVM-MCA checks if TAFFO improves the execution of the loops by reducing the total amount of machine cycles required.

\section{Design and Implementation}
The tool is divided into two different parts: the Loops Finder and the LLVM-MCA analysis.

\subsection{Loop Extractor}
By using the LLVM API, I was able to write an LLVM Pass that searches loops in code and add an Assembly inline comment in the header and the exit block.\\
The Assembly comments are "LLVM-MCA-BEGIN" and "LLVM-MCA-END". They are used in the LLVM-MCA analysis.
In the event of nested loops, the pass marks only the most inner one.\\
Before running this pass, we use the loop simplify pass to try to have simple form loops, with a pre-header and just one exit block.\\
OTP executes the Loop Finder Pass after the TAFFO ones.\\

\subsection{LLVM-MCA analysis}

I had to modify the main of LLV-MCA so that he shows the total machine cycle count for all marked regions. LLVM-MCA is designed to print several different analysis concerned the different region that was marked in the code. For this project, I was interested only in the total amount of machine cycle for all the regions.
In the file "llvm-mca-mod.c" I underline the originally LLVM-MCA code and the few lines that I change.

\subsection{Use a heuristic approach to different weight the loops}
WIP.


\subsection{Tool Steps}
The tool follows thees main step:
\begin{enumerate}
\item By using CLANG, it compiles the code and emits the LLVM IR.
\item To generate the TAFFO IR file, OPT runs all TAFFO passes, and then the Loop Extractor pass.
\item For the non TAFFO IR file, it executes only Loop Extractor pass.
\item The two IR files are compiled using LLC.
\item Uses the mod version of MCA on both the assembly.
\item Print the results of the analysis for each of the two assembly file.
\end{enumerate}

\section{Experimental Results (WIP)}
I run the tool on several tests that are used for TAFFO.

\subsection{Polybench benchmarks}
The Polybench benchmarks are 30 different tests that have one loop that dominates all the execution times.
I run tests and get the MCA analysis for two different processors: an INTEL-I7 x86-64 CPU processor and an INTEL-CORE2 x86-64.

\begin{center}
\begin{tabular}{ |p{7cm} | p{4cm} | p{4cm}| }
 \hline
 \multicolumn{3}{|c|}{Result of LLVM-MCA} \\
 \hline
   & INTEL X86 CORE2 & INTEL X86 I7\\
 \hline
  Total number of tests improved by TAFFO & 10 & 20 \\
  \hline
  Average of machine cycle reduced by TAFFO & WIP & WIP \\
  \hline

\end{tabular}
\end{center}


\section{Conclusions}

WIP

\end{document}